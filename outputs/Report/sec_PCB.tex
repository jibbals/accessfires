\section{Pyrocumulonimbus}
  \label{pcb}
  
  \paragraph{Formation}
    General hand wavey physics etc.
     
  \paragraph{Model formation}
    Details regarding how PCB is found/examined in model output.
    Story about how the parameters affect model PCB formation.
    Some stuff about vorticity and other metrics?
    
  \subsection{Waroona}
    \label{pcb:waroona}
    
    Figure \ref{fig:pcb:waroona:vert_motion_slices} shows horizontal model slices (at varying altitudes) of vertical wind motion (filled contour map), and cloud content (black outlines). 
    
    \begin{figure}
      \includegraphics[width=\linewidth]{../../figures/waroona_run3/vert_motion_slices/fig_201601060630.png}
      \caption{%
        Top-down views of vertical motion on model levels of increasing altitude from left to right, top to bottom. 
        A red contour shows the fire front, and cloud content above the 0.1 g\/kg threshold is marked by stippling.
        }
      \label{fig:pcb:waroona:vert_motion_slices}
    \end{figure}
    
    
  \paragraph{PCB Formation Threshold (PFT)}
    Kevin's PFT - point to his publication.
    Calculation within model output
    Summary showing that when fire spreads fast the output is higher.
    High output coinciding with low PFT is exactly when PCB occurs.
  