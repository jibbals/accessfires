\section{Pyrocumulonimbus}
  \label{pcb}
  
  Pyrocumulonimbus (PCB) is the phenomenon whereby a large thunderstorm system is generated by the heat and moisture entrained and lofted by a fire.
  These are extremely dangerous, difficult to predict, drastically change local weather, and can lead to substantial spotting and lightning \cite{Peace2017}.
  
  \paragraph{PCB Formation} is driven by the massive amounts of heat energy output from fires, which can be entrained in large scale updrafts.
  When a fire spreads fast is when the most heat is being produced, which is when the risk of PCB formation is greatest.
  Near-surface atmospheric stability and dynamics affects both how the heat entrains into the atmosphere and how fast the fire spreads.
  This means that simulated PCB are quite sensitive to the sorts of atmospheric parameters that affect horizontal and vertical movement, and near-surface stability.
  TODO: Some stuff about vorticity and other metrics?
  
  \paragraph{PCB Formation Threshold (PFT)}
  Kevin's PFT - TODO: point to his publication.
  Weather conditions can be more or less suitable for PCB generation, one way of quantifying this is through the PFT that estimates how much energy needs to be released by a fire to create a PCB.
  The less energy required, the more likely we are to see PCB formation.
  PFT calculations can be heavily influenced by energy in the system coming from the coupled fire model, making PFT analysis unsuitable downwind of the fire.
  Analysis on PFT in this work therefor either uses PFT calculated somewhere upwind of the fire, or meteorological output from an uncoupled (but otherwise identical) model run.
  
  \subsection{Waroona}
    \label{pcb:waroona}
    
    PFT from code developed by Kevin Tory can tell us whether PCB formation is likely.
    Fire output is compared to PFT calculated just upwind of the fire ignition point in figure \ref{fig:pcb:waroona:PFT}.
    Fire power at several times exceeds the PFT, and we might expect to see PCB within the model output.
    This is not a sure thing as PFT calculations can vary greatly over a small distance, and upwind conditions may not represent those at the firefront.
    
    \begin{figure}
      \includegraphics[width=\linewidth]{../../figures/waroona_run3/PFT_work/comparison/firepower.png}
      \caption{%
        Fire power output, and PFT calculated approximately 1~km upwind from the fire ignition for the Waroona fire simulation.
      }
      \label{fig:pcb:waroona:PFT}
    \end{figure}
    
    
    Strong cylindrical upward motion and cloud formation over the fire front can be considered clear evidence of PCB.
    To home in on where PCB occur, I zoom to a subset of the horizontal region that surrounds the fire. Horizontal model slices (at varying altitudes) of vertical wind motion and total cloud content show wind and cloud structures above the fire zone (Figure \ref{fig:pcb:waroona:vert_motion_slices}).
    A clear cylindrical plume occurs from around 5~km altitude all the way up to the stratosphere ($\sim 15$~km).
    
    \begin{figure}
      \includegraphics[width=\linewidth]{../../figures/waroona_run3/vert_motion_slices/fig_201601060630.png}
      \caption{%
        Top-down views of vertical motion on model levels of increasing altitude from left to right, top to bottom. 
        Grey dots represent Waroona, and Yarloop (north, south respectively), with a blue star showing a weather station site.
        A red contour shows the fire front, and cloud content above the 0.1 g\/kg threshold is marked by stippling.
        }
      \label{fig:pcb:waroona:vert_motion_slices}
    \end{figure}
    

    

  
  \subsection{Sir Ivan}
    \label{pcb:sirivan}
    