\section{Ember storm}
  \label{emberstorm}

  The Waroona fire was exacerbated by emberstorms that led to rapid downwind spotting.
  Spotting and ember transport is not built into the fire spread model, but atmospheric conditions do allow analysis of the contemporary meteorology.
  Waroona exhibited two emberstorms, the first one occurred on the first evening when the fire spread rapidly down to the base of the escarpment.
  The second one occurred on the second day (TODO: find out more about this one).
  
  The first emberstorm appears to be driven by downslope winds and a jump in vertical motion at the base of the escarpment.
  Figure \ref{fig:emberstorm:jump} shows the top-down and transectional views of winds, potential temperature, and fire front location at 10pm local time.
  The first feature of note is that along the nearby escarpment, rapid downward motion can be seen everywhere (blue dashed lines) except for directly in front of the fire.
  The transectional view shows that right as the fire spreads down the escarpment the vertical motion and wind speeds appear suited to lifting embers.
  This period of the model run exhibits rapid westward fire spread, which could lead to embers being entrained into the hot rising smoke plume.
  \begin{figure}
    \includegraphics[width=\linewidth]{../../figures/waroona_run3/emberstorm/transect0/fig_201601061400.png}
    \caption{%
      Top panel: Top down view of topography, surface horizontal winds (quivers), and 300m above ground vertical wind contour. Vertical winds are shown with dashed lines (at 3 m/s) upwards and downwards (pink, blue respectively). 
      Waroona is marked by a grey dot.
      Middle panel: Potential temperature transect along dashed black line in panel 1. An arrow points to the X-axis depicting the west-most point of the fire front, and an arrow along the Y-axis showing the altitude of the vertical wind contours shown in the top panel.
      Bottom panel: Vertical motion transect (coloured) with quivers showing wind speed and direction on the east-west-vertical plane.}
    \label{fig:emberstorm:jump}
  \end{figure}