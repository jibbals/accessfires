% Based on simple latex report template from %http://www.cs.technion.ac.il/~yogi/Courses/CS-Scientific-Writing/examples/simple/simple.htm

\title{ACCESS-Fire report}

\author{
        Jesse Greenslade\\
          \and
        Mika Peace\\
        Bureau of Meteorology\\
          \and
        Harvey Ye\\
          \and
        Jeff Kepert\\
          \and
        Kevin Tory\\
}
\date{\today}

\documentclass[12pt]{article}

\begin{document}
\maketitle
\tableofcontents


\begin{abstract}
This is the abstract \ldots
\end{abstract}

\section{Introduction} 
  \label{intro}

  Introduction to project and model

  \paragraph{Outline}
    The remainder of this article is organized as follows...
  
\section{Model} 
  \label{model}

\section{Pyrocumulonimbus}
  \label{pcb}
  
  \paragraph{Formation}
    General hand wavey physics etc.
     
  \paragraph{Model formation}
    Details regarding how PCB is found/examined in model output.
    Story about how the parameters affect model PCB formation.
  
    Some stuff about vorticity and other metrics?
    
  \paragraph{PCB Formation Threshold}
    Kevin's PFT - point to his publication.
    Calculation within model output
    Summary showing that when fire spreads fast the output is higher.
    High output coinciding with low PFT is exactly when PCB occurs.
  
  

\section{Ember storm}
  \label{emberstorm}


\bibliographystyle{abbrv}
\bibliography{firebib}

\end{document}
This is never printed
